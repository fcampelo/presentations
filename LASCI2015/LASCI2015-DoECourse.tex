\documentclass[t]{beamer}

% Load general definitions
% Preamble file - general definitions, package loading, etc.

%=================================
% Load packages
\usepackage{amssymb,amsmath}
\usepackage{graphicx}
\usepackage{url}
\usepackage{tikz}
\usetikzlibrary{mindmap,trees,arrows}
\usepackage{fancyvrb}
\usepackage[english]{babel}
\usepackage{times}
\usepackage[T1]{fontenc}
\usepackage[latin1]{inputenc}
\usepackage{subfigure}
\usepackage{times}
\usepackage[T1]{fontenc}
\usepackage{cancel}
\usepackage{color}
\usepackage{listings}
\usepackage{subfigure}

%=================================
% Set mode
\mode<presentation>
{
	\usetheme{CambridgeUS}
	\usecolortheme{rose}
	\useoutertheme{infolines}
	\setbeamercovered{invisible}
}

% Get rid of nav bar
\beamertemplatenavigationsymbolsempty

% Insert frame number at bottom of the page.
\usefoottemplate{\hfil\tiny{\color{black!90}\insertframenumber}} 

%=================================
% Define new commands

\newcommand\Real{{\mathbb{R}}}
%\newcommand{\vi}{\vspace{0.6\baselineskip}}
%\newcommand{\goodgap}{\hspace{\subfigtopskip}\hspace{\subfigbottomskip}}


% Equation environments
\newcommand{\beq}{\begin{equation}}
\newcommand{\eq}{\end{equation}}
\newcommand{\beqs}{\begin{equation*}}
\newcommand{\eqs}{\end{equation*}}
\newcommand{\beqn}{\begin{eqnarray}}
\newcommand{\eqn}{\end{eqnarray}}

% Bold variables
\newcommand{\mbf}[1]{\ensuremath{\mathbf{#1}}}

% Itemization
\newcommand{\bitem}{\begin{itemize}}
\newcommand{\eitem}{\end{itemize}}
\newcommand{\spitem}{\vskip 1em\item}
\newcommand{\bitems}{\begin{itemize}\item}
\newcommand{\benums}{\begin{enumerate}\item}
\newcommand{\eenum}{\end{enumerate}}

% color blocks
\newenvironment{colorblock}[2]{%
\setbeamercolor{block title}{#2}
\begin{block}{#1}}{\end{block}}

% Vertical spacing
\newcommand{\vone}{\vskip 1em}
\newcommand{\vhalf}{\vskip .5em}

% Frame environments
\newenvironment{ftst}[3][t]{%
\begin{frame}{environment=ftst,#1}
\frametitle{#2}
\framesubtitle{#3}}{\end{frame}}

\newenvironment{ftstf}[2]{
\begin{frame}[fragile,environment=ftstf]
\frametitle{#1}
\framesubtitle{#2}}{\end{frame}}

% colors
\definecolor{MyGray}{rgb}{0.5,0.5,0.5}
\definecolor{MyDBGray}{rgb}{0.1,0.1,0.4}
\definecolor{darkgreen}{rgb}{0,0.4,0}
\definecolor{black}{rgb}{0,0,0}
\def\defn#1{{\color{red} #1}}

% Footnote
\renewcommand{\thefootnote}{\alph{footnote}}

% Relaxed footnotes
\newcommand{\lfr}[1]{\let\thefootnote\relax\footnote{\tiny #1}}

% Verbatim environment - using FANCYVRB package
\DefineVerbatimEnvironment%
{rcode}{Verbatim}
{fontsize=\scriptsize}

% Verbatim environment - using LISTINGS package
%\lstnewenvironment{rcode} {\lstset{	language = R,
%									basicstyle = \scriptsize\ttfamily,
%									showspaces = false,
%									showstringspaces = false,
%									showtabs = false,
%									keywordstyle = \color{black}\bfseries,
%									commentstyle = \color{darkgreen},
%									numbers = none,
%									otherkeywords={	<-,
%													ggplot,
%													geom_boxplot,
%													facet_grid,
%													shapiro.test,
%													fligner.test,
%													glht,
%													with},
%									deletekeywords={data,
%													model,
%													residuals,
%													c,
%													axis,
%													default,
%													labels,
%													qq.text}}}%
%{}


% Specific definitions
\title[]{Design of Experiments}
\subtitle[]{and Experimental Comparison of Evolutionary Algorithms}
\author[]{Prof. Felipe Campelo, Ph.D.}
\institute{Dept. Electrical Engineering, Universidade Federal de Minas Gerais}
\date{Curitiba, Brazil - October 2015}

\begin{document}

% cover page
\setbeamertemplate{footline}{}
\begin{frame}
\vskip 5em
  \titlepage
  \begin{tikzpicture}[remember picture,overlay]
  \node[anchor=south west,yshift=5pt] at (current page.south west) {\includegraphics[width=.15\textwidth]{../figures/by-nc-sa.png}};
  \node[anchor=north east,yshift=-10pt,xshift=-10pt] at (current page.north east) {\includegraphics[width=.25\textwidth]{../figures/principal_completa3_ufmg.jpg}};
  \node[anchor=north west,yshift=-15pt,xshift=10pt] at (current page.north west) {\includegraphics[width=.3\textwidth]{../figures/ORCSBanner.png}};
  \end{tikzpicture}%  
\end{frame}

%=====

% Opening page
\begin{frame}
\frametitle{Main Concepts}
\begin{tikzpicture}[remember picture,overlay]
    \node[anchor=east,yshift=-15pt,xshift=0pt] at (current page.east)
    {\includegraphics[width=.8\textwidth]{../figures/mindmap.png}};
\end{tikzpicture}%
\end{frame}

%=====
% Main slides
\begin{ftst}
{Motivation}
{Experimental analysis of algorithms}
J.N. Hooker (1994): ``\textit{In other words, we should try to build an empirical science of algorithms}'';
\vone
Theoretical analyses are elegant, but sometimes inadequate, impractical, or unrepresentative;
\vone
Experimental investigation can provide valuable information - \textbf{as long as it is done properly!}
\vone
\begin{block}{}
\centering\textbf{Ideal situation: theory + experimentation}
\end{block}
\begin{tikzpicture}[remember picture,overlay]
\node[anchor=north east,yshift=-8pt,xshift=0pt] at (current page.north east) {\includegraphics[width=0.12\textwidth]{../figures/mindmap-small.png}};
\end{tikzpicture}%
\end{ftst}

%=====

\begin{ftst}
{Motivation}
{Main question}
\vskip 3em
\begin{block}{}
\Large\centering\textit{How much of the observed difference in performance between algorithms is due to actual differences in behavior, and how much is random noise?}
\end{block}
\begin{tikzpicture}[remember picture,overlay]
\node[anchor=north east,yshift=-8pt,xshift=0pt] at (current page.north east) {\includegraphics[width=0.12\textwidth]{../figures/mindmap-small.png}};
\end{tikzpicture}%
\end{ftst}

%=====

\begin{ftst}
{Motivation}
{Careless experimentation}
\begin{columns}[c]
\column[T]{0.75\textwidth}
\begin{block}{}
\begin{flushright}
``\textit{To consult the statistician after an experiment is\\
finished is often merely to ask him to conduct a post\\
mortem examination. He can perhaps say what\\
the experiment died of.}''\\
{\scriptsize -- Sir Ronald Fisher}
\end{flushright}
\end{block}
\column[T]{0.45\textwidth}\begin{center}
\end{center}
\end{columns}
\vone\vone
Unfortunately still common in the EA literature, but top journals are increasingly requiring higher methodological standards.
\vone
Tends to generate strongly biased results in favor of the ``proposed method'' (unsurprisingly).
\begin{tikzpicture}[remember picture,overlay]
\node[anchor=north east,yshift=-8pt,xshift=0pt] at (current page.north east) {\includegraphics[width=0.12\textwidth]{../figures/mindmap-small.png}};
\node[anchor=north east,yshift=-69pt,xshift=-30pt] at (current page.north east) 
{\includegraphics[height=2.4cm]{../figures/fisher.png}};
\end{tikzpicture}%
\lfr{Image: \url{http://www.nndb.com/people/763/000196175/}}
\end{ftst}

%=====

\begin{ftst}
{Motivation}
{Careless experimentation}
\begin{tikzpicture}[remember picture,overlay]
\node[anchor=north east,yshift=-8pt,xshift=0pt] at (current page.north east) {\includegraphics[width=0.12\textwidth]{../figures/mindmap-small.png}};
\end{tikzpicture}%
\begin{columns}
\column[t]{0.5\textwidth}
\textbf{Proposed method}
\vhalf
\bitems Careful implementation and debugging;
\spitem Exhaustive tuning;
\spitem Many experimental runs;
\eitem

\column[t]{0.5\textwidth}
\textbf{Competing algorithm}
\vhalf
\bitems Careless implementation;
\spitem Use of ``literature values'' for parameters;
\spitem Single experimental run;
\eitem
\end{columns}
\vone\vone\pause
\begin{block}{}
\centering\textbf{Description in the manuscript}\\
\centering``\textit{The values presented here are the result of 30 independent runs of the algorithms}''.
\end{block}
\end{ftst}

%=====

\begin{ftst}
{Motivation}
{Some other common issues}
Lack of a clear definition of the question one is trying to answer and of the hypotheses one is trying to test;
\vone
Inversion of the experimental rationale: desperate search to \textbf{demonstrate that my algorithm is the best}, instead of \textbf{investigating whether} it really is;
\vone
Lack of reproducibility.
\begin{tikzpicture}[remember picture,overlay]
\node[anchor=north east,yshift=-8pt,xshift=0pt] at (current page.north east) {\includegraphics[width=0.12\textwidth]{../figures/mindmap-small.png}};
\end{tikzpicture}%
\end{ftst}

%=====

\begin{ftst}
{Motivation}
{Other common issues}
Lack of a clear definition of the question one is trying to answer and of the hypotheses one is trying to test;
\vone
Inversion of the experimental rationale: desperate search to \textbf{demonstrate that my algorithm is the best}, instead of \textbf{investigating whether} it really is;
\vone
Lack of reproducibility.
\begin{tikzpicture}[remember picture,overlay]
\node[anchor=north east,yshift=-8pt,xshift=0pt] at (current page.north east) {\includegraphics[width=0.12\textwidth]{../figures/mindmap-small.png}};
\node[anchor=south,yshift=40pt,xshift=0pt] at (current page.south) 
{\includegraphics[width=.15\textwidth]{../figures/dontpanic.png}};
\end{tikzpicture}%
\lfr{Image: http://vigilantmeadow.deviantart.com/art/DON-T-PANIC-165415311}
\end{ftst}

%=====

\begin{ftst}
{Good practices}
{Important concepts in experimental algorithmics}
\begin{columns}
\column[T]{0.48\textwidth}
\bitems Relevant experiments;
\spitem Placement within the literature;
\spitem Adequate test instances;
\spitem Good experimental desing;
\spitem Efficient implementations;
\eitem
\column[T]{0.48\textwidth}
\bitems Reproducibility;
\spitem Comparability;
\spitem Telling the whole story;
\spitem Support for the conclusions;
\spitem Correct presentation of results;
\eitem
\end{columns}
\begin{tikzpicture}[remember picture,overlay]
\node[anchor=north east,yshift=-8pt,xshift=0pt] at (current page.north east) {\includegraphics[width=0.12\textwidth]{../figures/mindmap-small.png}};
\end{tikzpicture}%
\lfr{Based on D.S. Johnson: \textit{A Theoretician's Guide to the Experimental Analysis of Algorithms} - \url{http://plato.asu.edu/ftp/experguide.pdf}}
\end{ftst}

%=====

\begin{ftst}
{Pre-experimental checklist}
{Before you even start...}
\begin{tikzpicture} [remember picture,overlay]
\node[anchor=north east,yshift=-8pt,xshift=0pt] at (current page.north east) {\includegraphics[width=0.12\textwidth]{../figures/mindmap-small.png}};
\end{tikzpicture}%
Is the comparison relevant?
\vone
Will the (possible) results be of interest to anyone?
\vone 
Does it have any practical implications?
\vone 
How does it fit within the literature?
\vone\vone\vone
\end{ftst}

%=====

\begin{ftst}
{Pre-experimental checklist}
{Before you even start...}
\begin{tikzpicture} [remember picture,overlay]
\node[anchor=north east,yshift=-8pt,xshift=0pt] at (current page.north east) {\includegraphics[width=0.12\textwidth]{../figures/mindmap-small.png}};
\end{tikzpicture}%
Is the comparison relevant?
\vone
Will the (possible) results be of interest to anyone?
\vone 
Does it have any practical implications?
\vone 
How does it fit within the literature?
\vone\vone\vone
However...
\vhalf
{\scriptsize``\textit{Sometimes one should do a completely wild experiment,\\like blowing the trumpet to the tulips every morning\\for a month. Probably nothing would happen, but what if it did?}''\\}
{\tiny -- Sir George Howard Darwin}
\begin{tikzpicture} [remember picture,overlay]
\node[anchor=south east,yshift=-10pt,xshift=13pt] at (current page.south east) {\includegraphics[height=2.5cm]{../figures/ttt.png}};
\end{tikzpicture}%
\lfr{Girl playing: \url{http://www.film.queensu.ca/tulips/default.html}}
\end{ftst}

%=====

\begin{ftst}
{Pre-experimental design}
{The quest for \textit{The Question}}
\begin{columns}
\column[T]{0.9\textwidth}
\vskip 1em
\begin{block}{}
{\small \flushright``\textit{- Would you tell me, please, which way I ought to go from here?\\
- That depends a good deal on where you want to get to, said the Cat.\\
- I don't much care where - said Alice.\\
- Then it doesn't matter which way you go, the Cat replied}\\
\vhalf
Lewis Carroll, \textbf{Alice in Wonderland}\\}
\end{block}
\column[T]{0.3\textwidth}
\end{columns}
\vone\vone
The first (and possibly the most important) thing to determine is what exactly your experiment is intended to reveal / discover.

\begin{tikzpicture} [remember picture,overlay]
\node[anchor=north east,yshift=-8pt,xshift=0pt] at (current page.north east) {\includegraphics[width=0.12\textwidth]{../figures/mindmap-small.png}};
\node[anchor=north west,yshift=-105pt,xshift=0pt] at (current page.north west) 
{\includegraphics[width=0.12\textwidth]{../figures/cat.png}};
\end{tikzpicture}%
\lfr{Image adapted from \url{https://www.pinterest.com/pin/381609768395834338/}}
\end{ftst}

%=====

\begin{ftst}
{Pre-experimental design}
{The quest for \textit{The Question}}
\begin{tikzpicture} [remember picture,overlay]
\node[anchor=north east,yshift=-8pt,xshift=0pt] at (current page.north east) {\includegraphics[width=0.12\textwidth]{../figures/mindmap-small.png}};
\end{tikzpicture}%
What is the purpose of your experiment?
\begin{block}{}
\centering``\textit{I want to prove that my method is good!}''
\end{block}
\vone
Lets drop the exclamation mark and refine that a little, shall we?
\pause
\begin{block}{}
\centering``\textit{I want to \textbf{discover if} my method is good.}''
\end{block}
\vone
That's much better. \pause Now, ``good'' compared to what?
\begin{block}{}
\centering``\textit{I want to discover if my method is \textbf{better than algorithm A}.}''
\end{block}
\vone 
We're making progress...
\end{ftst}

%=====

\begin{ftst}
{Pre-experimental design}
{The quest for \textit{The Question}}
\begin{tikzpicture} [remember picture,overlay]
\node[anchor=north east,yshift=-8pt,xshift=0pt] at (current page.north east) {\includegraphics[width=0.12\textwidth]{../figures/mindmap-small.png}};
\end{tikzpicture}%
Lets proceed with our quest for \textit{The Question}: you want to compare your method to another, but on what exactly?
\pause
\begin{block}{}
\centering``\textit{I want to discover if my method is better than algorithm A \textbf{in terms of solution quality}}.''
\end{block}
\vone\pause
And how are you going to measure solution quality?
\pause
\begin{block}{}
\centering``\textit{I want to discover if my method is better than algorithm A in terms of solution quality, \textbf{measured using indicator $\mathcal{F}$}}.''
\end{block}
\end{ftst}

%=====

\begin{ftst}
{Pre-experimental design}
{The quest for \textit{The Question}}
\begin{tikzpicture} [remember picture,overlay]
\node[anchor=north east,yshift=-8pt,xshift=0pt] at (current page.north east) {\includegraphics[width=0.12\textwidth]{../figures/mindmap-small.png}};
\end{tikzpicture}%
Lets keep it up just a little more, we're almost there. When you say ``better'', what exactly do you mean? Typical, best, worst case?
\pause
\begin{block}{}
\centering``\textit{I want to discover if my method is better than algorithm A in terms of \textbf{average} solution quality, measured using \textbf{the mean of} indicator $\mathcal{F}$}.''
\end{block}
\vone
\pause
And you want to investigate if your method is good for what?
\pause
\begin{block}{}
\centering``\textit{I want to discover if my method is better than algorithm A, in terms of average solution quality (measured using the mean of indicator $\mathcal{F}$), \textbf{for the solution of a class of problems $\mathcal{Q}$}}.''
\end{block}
\end{ftst}

%=====

\begin{ftst}
{Pre-experimental design}
{The quest for \textit{The Question}}
\begin{tikzpicture} [remember picture,overlay]
\node[anchor=north east,yshift=-8pt,xshift=0pt] at (current page.north east) {\includegraphics[width=0.12\textwidth]{../figures/mindmap-small.png}};
\end{tikzpicture}%
The process of determining \textit{The Question} is an important one, which is often overlooked in experimental algorithmics.
\vone
Besides escaping the mockery of the Cheshire Cat, there are good reasons not to ignore this step:

\bitems \textit{HARKing} tends to greatly increase the rate of false positives in favor of the ``proposed approach'';
\spitem Thinking about \textit{The Question} forces the experimenter to consider important aspects of his or her research, such as scope and performance measurement;
\eitem
\lfr{Harking = \textit{Hypothesizing After the Results are Known}}
\end{ftst}

%=====

\begin{ftst}
{Pre-experimental design}
{Selection of test instances}
\begin{tikzpicture} [remember picture,overlay]
\node[anchor=north east,yshift=-8pt,xshift=0pt] at (current page.north east) {\includegraphics[width=0.12\textwidth]{../figures/mindmap-small.png}};
\end{tikzpicture}%%
The name of the game is \textit{Representativeness}!
\vone
Benchmark sets can result in \textit{overfitting} of the algorithms to specific instances;
\vhalf
Randomly generated instances may not be representative of real problems;
\vhalf
Arbitrary problem selection can introduce experimenter biases;
\vone
\begin{block}{Some ideas}
\bitems Random sampling of representative instances (\textit{random factor} approach)
\spitem Training and validation (\textit{machine learning} approach)\eitem
\end{block}
\end{ftst}

%=====

\begin{ftst}
{Pre-experimental design}
{Non-experimental parameters}
\begin{tikzpicture} [remember picture,overlay]
\node[anchor=north east,yshift=-8pt,xshift=0pt] at (current page.north east) {\includegraphics[width=0.12\textwidth]{../figures/mindmap-small.png}};
\end{tikzpicture}%%
How to deal with non-experimental parameters?

\bitems Fixed values? (generalization problem)
\spitem Randomized values? (representativeness problem) 
\spitem Literature values? (adequacy problem)
\eitem
\vhalf
\begin{block}{Tuning approach}
\bitems Use a portion of the computational budget of the experiment to tune these parameters;
\spitem Balanced effort for \textit{all} algorithms.
\eitem
\end{block}
\end{ftst}

%=====

\begin{ftst}
{Design of Experiments}
{What is it?}
\begin{tikzpicture} [remember picture,overlay]
\node[anchor=north east,yshift=-8pt,xshift=0pt] at (current page.north east) {\includegraphics[width=0.12\textwidth]{../figures/mindmap-small.png}};
\end{tikzpicture}%%
\vone\vone
\begin{block}{}
\centering \textit{Definition of data collection protocols that enable a correct analysis by means of statistical tools capable of supporting sound and objective conclusions.}
\end{block}
\vone
Necessary for conclusions to have some quantifiable \textit{meaning};
\vone
Useful for preventing mistakes due to personal biases or other experimental artifacts.
\end{ftst}

%=====

\begin{ftst}
{Design of Experiments}
{Sample size, replication and pseudoreplication - a short detour}
\begin{tikzpicture} [remember picture,overlay]
\node[anchor=north east,yshift=-8pt,xshift=0pt] at (current page.north east) {\includegraphics[width=0.12\textwidth]{../figures/mindmap-small.png}};
\node[anchor=south east,yshift=0pt,xshift=0pt] at (current page.south east) {\includegraphics[width=0.15\textwidth]{../figures/itt.png}};
\end{tikzpicture}%%
Suppose that we want to investigate the question: ``\textit{Is the average hair length different between students and professors in the Computational Intelligence field?}''
\vone
Lets assume that the audience of this tutorial is a representative sample of our population of interest;
\vone
If we take 5 professors and 5 students from the audience and measure 1 hair from each head, what is our sample size?
\vone\pause
What if we take 30 hairs from each head?
\lfr{Example adapted from B. Shipley, \textit{Cause and Correlation in Biology}. Cambridge University Press, 2000.}
\lfr{Cousin Itt: \url{http://kawiku.deviantart.com/art/Cousin-Itt-334962933}}
\end{ftst}

%=====

\begin{ftst}
{Design of Experiments}
{Sample size, replication and pseudoreplication - a short detour}
\begin{tikzpicture} [remember picture,overlay]
\node[anchor=north east,yshift=-8pt,xshift=0pt] at (current page.north east) {\includegraphics[width=0.12\textwidth]{../figures/mindmap-small.png}};
\node[anchor=south east,yshift=0pt,xshift=0pt] at (current page.south east) {\includegraphics[width=0.15\textwidth]{../figures/itt.png}};
\end{tikzpicture}%%
Sampling more hairs from a given head improves the precision of our estimate for that particular head, but \textit{\textbf{it does not increase our effective sample size!}}
\vone
In this example, performing statistical tests considering the 300 individual measurements (10 heads, 30 hairs per head) as independent values would falsely inflate our degrees of freedom.
\vone
This common mistake is called \textit{pseudoreplication}, and results in a much higher rate of false positives in statistical tests.
\vone
A simple solution is to use the average hair length per head\\
as the individual data points.
\lfr{Cousin Itt: \url{http://kawiku.deviantart.com/art/Cousin-Itt-334962933}}
\end{ftst}

%=====

\begin{ftst}
{Design of Experiments}
{Sample size, replication and pseudoreplication - a short detour}
\begin{tikzpicture} [remember picture,overlay]
\node[anchor=north east,yshift=-8pt,xshift=0pt] at (current page.north east) {\includegraphics[width=0.12\textwidth]{../figures/mindmap-small.png}};
\end{tikzpicture}%%
The analogy between the hair example and the comparison of EAs is quite straightforward:
\begin{block}{}
\begin{center}
\begin{tabular}{l|c|c}
\footnotesize
                                    & \textbf{Hair example}        & \textbf{Algorithm comparisons} \\ 
\hline
Population of interest    & People in CI                       & Problem class $\mathcal{Q}$\\
Comparison                  & Student $\times$ Prof.       & Alg. A $\times$ Alg. B\\ 
Observation unit           & Head                                & Problem instance\\ 
Within-unit replicate     & Individual hairs                  & Individual runs\\
\end{tabular} 
\end{center}
\end{block}
\vhalf
More runs will buy you \textit{some} power by reducing the uncertainties associated with each instance. 
\vhalf
For greater power, \textit{more instances} is the way to go.
\lfr{Check M. Birattari's \textit{On the estimation of the expected performance of a metaheuristic on a class of instances. How many instances, how many runs?} (\url{http://goo.gl/TVo6YI}) for a deeper discussion.}
\end{ftst}

%=====

\begin{ftst}
{Design of Experiments}
{Minimally relevant difference}
\begin{tikzpicture} [remember picture,overlay]
\node[anchor=north east,yshift=-8pt,xshift=0pt] at (current page.north east) {\includegraphics[width=0.12\textwidth]{../figures/mindmap-small.png}};
\end{tikzpicture}%%
More instances (and, to a certain extent, more runs per instance) will provide increased power in statistical comparisons. 
\vone
\textit{Just don't overdo it!}
\vone
To avoid falling victim to \textit{p-hacking}, it is important to define (prior to running the experiment) what is the smallest difference that would have any practical implication.
\vone
When analysing and describing the experiment, this practical threshold can provide a much needed reality check for the eager researcher.
\lfr{\textit{p-hacking}: tweaking your experiment (or torturing your data) until it gives you statistical significance.}
\end{ftst}

%=====

\begin{ftst}
{Design of Experiments}
{Statistical power - a(nother) short detour}
\begin{tikzpicture} [remember picture,overlay]
\node[anchor=north east,yshift=-8pt,xshift=0pt] at (current page.north east) {\includegraphics[width=0.12\textwidth]{../figures/mindmap-small.png}};
\end{tikzpicture}%%
The earlier discussion gave us some insight on \textit{how to think} about comparative experiments in evolutionary computation;
\vone
The power (i.e., sensitivity) of a given experiment to detect a certain difference in performance between two algorithms is a function of some factors:

\bitems \textit{Sample size} $\checkmark$\pause
\spitem Significance level  $\checkmark$\pause
\spitem Effect size  $\checkmark$\pause
\spitem Residual variance (i.e., unnacounted variability)
\eitem

\end{ftst}

%=====

\begin{ftst}
{Design of Experiments}
{Statistical power - a(nother) short detour}
\begin{tikzpicture} [remember picture,overlay]
\node[anchor=north east,yshift=-8pt,xshift=0pt] at (current page.north east) {\includegraphics[width=0.12\textwidth]{../figures/mindmap-small.png}};
\node[anchor=south west,yshift=25pt,xshift=-5pt] at (current page.south west) 
{\includegraphics[width=0.17\textwidth]{../figures/fuel.png}};
\end{tikzpicture}%%
%The larger the amount of variability that is not accounted in our models, the greater the risk that it may mask the effects of interest.
Suppose that you want to investigate the mean efficiency of different fuel mixtures (in terms of km/\$) for a fleet of vehicles.
\vone
Between-vehicle variation is a potentially large source of variability (possibly larger than the differences due to different fuels);
\vone
This variability can be modeled by considering each vehicle as an \textit{experimental unit}, and isolating its effects in the statistical model.
\vone
\begin{columns}
\column[T]{0.2\textwidth}
\column[T]{0.7\textwidth}
For known and controllable sources of spurious variation (such as the vehicles in this example) the technique used to model this variation out of our inference is called \textit{blocking}.
\end{columns}
\lfr{Image: \url{https://financialsensei.wordpress.com/tag/save-money-on-gas/}}
\end{ftst}

%=====

\begin{ftst}
{Design of Experiments}
{Blocking}
\begin{tikzpicture} [remember picture,overlay]
\node[anchor=north east,yshift=-8pt,xshift=0pt] at (current page.north east) {\includegraphics[width=0.12\textwidth]{../figures/mindmap-small.png}};
\end{tikzpicture}%%
\textit{Blocking} is the principle behind well-known techniques for comparing algorithms on multiple instances, such as \textit{Friedman's test} and \textit{Wilcoxon-Mann-Whitney test};
\vone
However, the usually shunned parametric counterparts (\textit{blocked ANOVA} and \textit{paired t-tests}) also deserve some attention;
\vone
The assumption of normal \textit{sampling distribution of the means} is generally well covered by the Central Limit Theorem even for modest sample sizes (important exceptions: \textbf{\textit{truncated observations}}, \textbf{\textit{extreme skewness}}, and \textbf{\textit{outliers}}).
\vone
Parametric methods usually present larger power (i.e., greater sensitivity) and are simpler to understand.
\end{ftst}

%=====

\begin{ftst}
{Design of Experiments}
{Central Limit Theorem}
\begin{tikzpicture} [remember picture,overlay]
\node[anchor=north east,yshift=-8pt,xshift=0pt] at (current page.north east) {\includegraphics[width=0.12\textwidth]{../figures/mindmap-small.png}};
\node[anchor=south,yshift=40pt,xshift=0pt] at (current page.south) 
{\includegraphics[width=\textwidth]{../figures/CLTdemo.png}};
\end{tikzpicture}%%
\lfr{Interactive demo: \url{http://drwho.cpdee.ufmg.br:3838/CLT/}}
\lfr{Source code: \url{http://git.io/vnPj8}}
\end{ftst}

%%===== FOR THE EICEFALA. ALSO INCLUDE DISCUSSION ON RANDOMIZATION.
%
%\begin{ftst}
%{Design of Experiments}
%{Follow the plan!}
%\begin{tikzpicture} [remember picture,overlay]
%\node[anchor=north east,yshift=-8pt,xshift=0pt] at (current page.north east) {\includegraphics[width=0.12\textwidth]{../figures/mindmap-small.png}};
%\end{tikzpicture}%%
%
%\end{ftst}

%=====

\begin{ftst}
{Statistical modeling and inference}
{A consequence of design}
\begin{tikzpicture} [remember picture,overlay]
\node[anchor=north east,yshift=-8pt,xshift=0pt] at (current page.north east) {\includegraphics[width=0.12\textwidth]{../figures/mindmap-small.png}};
\node[anchor=south east,yshift=10pt,xshift=-5pt] at (current page.south east) 
{\includegraphics[width=.2\textwidth]{../figures/Rlogo.png}};
\end{tikzpicture}%%
If the experiment is well designed, its planning essentially determines the statistical model to be used (at least qualitatively);
\vone
The analysis techniques are usually simple (but the devil is in the details);
\vone
Use of existing tools and techniques;
\vone
Inference on the \textit{statistical significance} of the results;
\lfr{R: A Language and Environment for Statistical Computing - \url{http://www.R-project.org/}}

\end{ftst}

%=====

\begin{ftst}
{Statistical modeling and inference}
{A consequence of design}
\begin{tikzpicture} [remember picture,overlay]
\node[anchor=north east,yshift=-8pt,xshift=0pt] at (current page.north east) {\includegraphics[width=0.12\textwidth]{../figures/mindmap-small.png}};
\node[anchor=south east,yshift=10pt,xshift=-5pt] at (current page.south east) 
{\includegraphics[width=.2\textwidth]{../figures/Rlogo.png}};
\end{tikzpicture}%%
If the experiment is well designed, its planning essentially determines the statistical model to be used (at least qualitatively);
\vone
The analysis techniques are usually simple (but the devil is in the details);
\vone
Use of existing tools and techniques;
\vone
Inference on the \textit{statistical significance} of the results;
\lfr{R: A Language and Environment for Statistical Computing - \url{http://www.R-project.org/}}

\end{ftst}

%=====

\begin{ftst}
{Reporting of results}
{Presentation}
\begin{tikzpicture} [remember picture,overlay]
\node[anchor=north east,yshift=-8pt,xshift=0pt] at (current page.north east) {\includegraphics[width=0.12\textwidth]{../figures/mindmap-small.png}};
\node[anchor=south east,yshift=100pt,xshift=-15pt] at (current page.south east) 
{\includegraphics[height=2.5cm]{../figures/tufte.jpg}};
\node[anchor=south east,yshift=20pt,xshift=-15pt] at (current page.south east) 
{\includegraphics[height=2.5cm]{../figures/yau.png}};
\end{tikzpicture}%%
Combine textual, numeric and graphical elements to tell a story with your data. It simplifies the understanding and analysis of the results.
\begin{columns}[T]
\column{0.75\textwidth}
	\bitems Strive to achieve graphical excellence;
		\spitem Coherence of notation - special attention to figures and tables;
		\spitem Display simultaneous confidence intervals and other graphical indicators of effect size.
	\eitem
\column{0.25\textwidth}
\end{columns}
\lfr{Other great resources on graphical excellence:}
\lfr{\textit{Flowing Data} (\url{http://flowingdata.com/})}
\lfr{\textit{Information is Beautiful} (\url{http://www.informationisbeautiful.net})}
\end{ftst}

%=====

\begin{ftst}
{Reporting of results}
{Tell the whole story}
Avoid \textit{cherrypicking} your results;
\vone
Report and describe anomalous results and outliers (even if they are discarded in the modeling phase);
\vone
Exercise \textit{extreme} caution when discarding outliers!
\vone
Detail stop criteria, computational cost, and any other relevant information for the understanding and reproducibility of your results.
\vone
Whenever possible, share your code! It's good for the field as a whole (for reproducibility), and it is good for your paper (for citations)!
\lfr{R.D. Peng, \textit{Reproducible Research in Computational Science}, Science 334(6060):1226-1227, 2011}
\lfr{P. Vandewalle, \textit{Code Sharing Is Associated with Research Impact in Image Processing}, Computer Science and Engineering 14(4):42-47, 2012}
\end{ftst}


%=====

\begin{ftst}
{Conclusions}
{Drawing and reporting conclusions}
\begin{tikzpicture} [remember picture,overlay]
\node[anchor=north east,yshift=-8pt,xshift=0pt] at (current page.north east) {\includegraphics[width=0.12\textwidth]{../figures/mindmap-small.png}};
\end{tikzpicture}%%
Conclusions should be based on solid evidence from the data;
\vone
Be conservative - don't exaggerate the generality of the results;
\vone
Report significance levels, effect sizes, and the assumptions under which the results are valid;
\vone
\textit{Suggest explanations} to the observed results;
\vone
Be careful with \textit{anomaly hunting};
\begin{columns}
\column[T]{\textwidth}
\begin{block}{}
	\flushright\small``\textit{Always let the science drive the statistics. If you get a statistically significant result, go back and describe what it means in the scientific context.}`''
	\flushright\small -- Aaron Rendahl
\end{block}
\column[T]{.1\textwidth}
\end{columns}
\end{ftst}

%=====

%=====

\begin{ftst}
{Want more?}
{Some shameless self-promotion}
\begin{tikzpicture} [remember picture,overlay]
\node[anchor=north east,yshift=-8pt,xshift=0pt] at (current page.north east) {\includegraphics[width=0.12\textwidth]{../figures/mindmap-small.png}};
\end{tikzpicture}%%
\vone
If you liked this short tutorial, check the online materials available at:

\vone\vone
{\footnotesize\centering\url{https://github.com/fcampelo/Design-and-Analysis-of-Experiments}}
\vone\vone

\begin{flushright}
It's free, and I think you'll like it too!
\end{flushright}
\end{ftst}

%=====

\begin{ftst}
{Questions?}
{\ }
\begin{tikzpicture} [remember picture,overlay]
\node[anchor=south,yshift=40pt] at (current page.south)
{ \includegraphics[width=.25\textwidth]{../figures/Itt.png}};
\node[anchor=south,yshift=130pt,xshift = 30] at (current page.south)
{ \includegraphics[angle = -20,width=.1\textwidth]{../figures/question.png}};
\node[anchor=north east,yshift=-8pt,xshift=0pt] at (current page.north east) {\includegraphics[width=0.12\textwidth]{../figures/mindmap-small.png}};

\end{tikzpicture}
\lfr{Images: \url{http://kawiku.deviantart.com/art/Cousin-Itt-334962933}}
\lfr{\ \ \ \ \ \ \ \ \ \ \ \ \ \  \url{https://commons.wikimedia.org/wiki/File:Question-mark-blackandwhite.png}}
\end{ftst}

%=====

\begin{ftstf}{About this material}{Conditions of use and referencing}
\centering\footnotesize This work is licensed under the Creative Commons CC BY-NC-SA 4.0 license\\(Attribution Non-Commercial Share Alike International License version 4.0).\\
\vhalf
\url{http://creativecommons.org/licenses/by-nc-sa/4.0/}\\
\vhalf
\footnotesize Please reference this work as:
\vskip 0.2em
\scriptsize \flushleft Felipe Campelo, \textit{Design of Experiments and Statistical Comparison of Evolutionary Agorithms}. Online: {\scriptsize\url{http://git.io/vZph7}}, Latin American School on Computational Intelligence, October 13, 2015, Curitiba, Brazil; Creative Commons BY-NC-SA 4.0.\\

\begin{Verbatim}[fontsize=\tiny]
    @Misc{Campelo2015-LASCI,
      title={Design of Experiments and Statistical Comparison of Evolutionary Agorithms},
      author={Felipe Campelo},
      howPublished={http://git.io/vZph7},
      year={2015},
      month={October 13},
      note={Latin American School on Computational Intelligence, 
            Curitiba, Brazil; Creative Commons BY-NC-SA 4.0.}}
\end{Verbatim}

\begin{tikzpicture} [remember picture,overlay]
\node[anchor=south,yshift=0pt] at (current page.south){ \includegraphics[width=.2\textwidth]{../figures/CCSomerights.png}};
\end{tikzpicture}%
\end{ftstf}


\end{document}